\subsection{Project overview}
The project aims to develop a web application that allows users to track movies and series they watch in a simple, centralized, and user-friendly interface. The idea comes from “Seb”, a series enthusiast who wants to better manage his watched episodes, know which platform each series is available on, and annotate personal notes while keeping track of progress.

\subsection{Project Objectives}
\begin{itemize}
    \item Let users track where they left off in each series.
    \item Centralize information across streaming platforms.
    \item Provide a clean, simple, and accessible interface.
    \item Include light social features (comments, favorites).
    \item Offer fun statistics like total episodes watched.
\end{itemize}

\subsection{Core Features}

\subsubsection{Series Management}
\begin{itemize}
    \item Add a series to the user's list.
    \item View series details (title, synopsis, actors, platforms).
    \item Mark watched episodes (e.g., "up to episode 5").
    \item Archive completed or paused series.
    \item Add to favorites (heart or star icon).
\end{itemize}

\subsubsection{Platform Information}
\begin{itemize}
    \item Display available streaming platforms (e.g., Netflix, Prime Video).
    \item Optional: link to the platform.
\end{itemize}

\subsubsection{Personal Journal}
\begin{itemize}
    \item Allow comments per episode or season.
    \item Personal history visible only to the user.
\end{itemize}

\subsubsection{Personal Statistics}
\begin{itemize}
    \item Counter of total episodes watched.
    \item (Optional) Estimated time spent watching.
\end{itemize}

\subsection{Target Audience}
\begin{itemize}
    \item Regular streaming users.
    \item People who want a simple way to organize their viewing history.
    \item People who want to look up a simple information about a series or movie.
    \item Non-tech-savvy users: the interface must be highly intuitive.
\end{itemize}

\subsection{Technical Constraints}
\begin{itemize}
    \item Responsive web app (PC, tablet, mobile).
    \item Lightweight user experience (no login required at start).
    \item Simple database (e.g., MySQL).
    \item Clean UI with soft colors and readable fonts (e.g., WCAG norms)
    \item Optional: multilingual support.
\end{itemize}

\subsection{UI/UX Requirements}
\begin{itemize}
    \item Smooth navigation with intuitive menus.
    \item Visible buttons (e.g., episode “seen” checkbox).
    \item Essential information highlighted.
    \item Simple icons for archiving, favorites, etc.
    \item Soft, non-aggressive color scheme.
\end{itemize}

\subsection{Possible Future Features}
\begin{itemize}
    \item Recommendation engine based on favorites.
    \item Share progression with friends (if allowed).
    \item Notifications for new episodes (if allowed).
    \item API sync with streaming services (if allowed).
\end{itemize}

\subsection{Resources and Timeline}

\subsubsection{Technologies}
\begin{itemize}
    \item Frontend: Frameworks (React).
    \item Backend: Frameworks PHP (Laravel).
    \item Database: MySQL.
\end{itemize}

\subsection{Deliverables}
\begin{itemize}
    \item For 07/05/2025: this document with the UI mockups.
    \item For 16/05/2025: a conceptual model of the database.
    \item For 10/06/2025: a working prototype with the main features.
\end{itemize}
