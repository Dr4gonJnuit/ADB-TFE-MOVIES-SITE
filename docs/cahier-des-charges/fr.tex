\subsection{Aperçu du Projet}
Le projet vise à développer une application web permettant aux utilisateurs de suivre les séries qu'ils regardent via une interface simple, centralisée et conviviale. L'idée vient de "Seb", un passionné de séries qui souhaite mieux gérer les épisodes qu'il a vus, savoir sur quelle plateforme chaque série est disponible, et annoter des notes personnelles tout en suivant sa progression.

\subsection{Objectifs du Projet}
\begin{itemize}
    \item Permettre aux utilisateurs de suivre où ils en sont dans chaque série.
    \item Centraliser les informations sur les plateformes de streaming.
    \item Fournir une interface propre, simple et accessible.
    \item Inclure des fonctionnalités sociales légères (commentaires, favoris).
    \item Offrir des statistiques amusantes comme le nombre total d'épisodes regardés.
\end{itemize}

\subsection{Fonctionnalités Principales}

\subsubsection{Gestion des Films/Séries}
\begin{itemize}
    \item Ajouter une série à la liste de l'utilisateur.
    \item Voir les détails d'une série (titre, synopsis, plateformes).
    \item Marquer les épisodes vus pour la série en cochant une case.
    \item Archiver les séries terminées ou mises en pause.
    \item Ajouter aux favoris (icône cœur).
\end{itemize}

\subsubsection{Informations sur les Plateformes}
\begin{itemize}
    \item Afficher les plateformes de streaming disponibles (par exemple, Netflix, Prime Video).
    \item \underline{Optionnel :} lien vers la plateforme.
\end{itemize}

\subsubsection{Comptes Utilisateurs}
\begin{itemize}
    \item Permettre aux utilisateurs de créer un compte pour sauvegarder leur progression et leurs préférences.
    \item Stockage sécurisé des mots de passe et gestion des comptes.
    \item Fonctionnalités basées sur le compte : synchronisation entre appareils, recommandations personnalisées.
    \item \underline{Optionnel :} Connexion sociale (par exemple, Google, Facebook).
\end{itemize}

\subsubsection{Journal Personnel}
\begin{itemize}
    \item Permettre des commentaires par épisode ou saison.
    \item Historique personnel visible uniquement par l'utilisateur.
\end{itemize}

\subsubsection{Statistiques Personnelles}
\begin{itemize}
    \item Compteur du nombre total d'épisodes regardés.
    \item Nombre de séries terminées par l'utilisateur.
    \item \underline{Optionnel :} Nombre de séries en cours.
    \item \underline{Optionnel :} Temps estimé passé à regarder.
\end{itemize}

\subsection{Public Cible}
\begin{itemize}
    \item Utilisateurs réguliers de plateformes de streaming.
    \item Personnes souhaitant un moyen simple d'organiser leur historique de visionnage.
    \item Personnes cherchant à consulter une information simple sur une série.
    \item Utilisateurs non technophiles : l'interface doit être très intuitive.
\end{itemize}

\subsection{Contraintes Techniques}
\begin{itemize}
    \item Application web responsive (PC, tablette, mobile).
    \item Expérience utilisateur légère (pas de connexion requise au départ).
    \item Backend simple (par exemple, SQLite ou MySQL).
    \item Interface propre avec des couleurs douces et des polices lisibles.
    \item \underline{Optionnel :} support multilingue.
\end{itemize}

\subsection{Exigences UI/UX}
\begin{itemize}
    \item Navigation fluide avec des menus intuitifs.
    \item Boutons visibles (par exemple, case à cocher "épisode vu").
    \item Informations essentielles mises en avant.
    \item Icônes simples pour l'archivage, les favoris, etc.
    \item Palette de couleurs douces et non agressives.
\end{itemize}

\subsection{Fonctionnalités Futures Possibles}
\begin{itemize}
    \item Gestion des films en plus des séries.
    \item Moteur de recommandations basé sur les favoris.
    \item Partage de progression avec des amis (si autorisé).
    \item Notifications pour les nouveaux épisodes (si autorisé).
    \item Synchronisation API avec les services de streaming (si autorisé).
    \item Application mobile.
\end{itemize}

\subsection{Ressources et Chronogramme}

\subsubsection{Technologies}
\begin{itemize}
    \item Frontend : Frameworks (React).
    \item Backend : Frameworks PHP (Laravel).
    \item Base de données : MySQL.
\end{itemize}

\subsection{Livrables}
\begin{itemize}
    \item Pour le 07/05/2025 : ce document avec les maquettes UI.
    \item Pour le 16/05/2025 : une base de données fonctionnelle.
    \item Pour le 21/05/2025 : ce document révisé avec les maquettes UI.
    \item Pour le 10/06/2025 : un prototype fonctionnel avec les principales fonctionnalités.
\end{itemize}
